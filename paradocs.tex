\documentclass[a4paper,12pt]{article}
\usepackage{csquotes}
\usepackage{hyperref}
\title{First Paragraphs of Technical Documents}
\author{Zac Dover}

\begin{document}
\maketitle

\section{Upgrading and Repairing PCs}
\underline{Upgrading and Repairing PCs} by Scott Mueller begins with a paragraph consisting of two sentences. The first sentence broadly refers to the history of the development of the personal computer. The second sentence declares an intention to examine a few of those developments in order to give the reader a basis for understanding that history.
\begin{displayquote}
Many discoveries and inventions have directly and indirectly contributed to the development of the PC and other personal computers as we know them today. Examining a few important developmental landmarks can help bring the entire picture into focus.
\end{displayquote}

\section{Red Hat Enterprise Virtualization 3.4 \\ Administration Guide}
This guide begins with a list of the components of a RHEV 3.4 environment.
\begin{displayquote}
{\tiny A Red Hat Enterprise Virtualization environment consists of:}
\begin{itemize}
{\tiny  \item Virtual machine \textbf{hosts} using the Kernel-based Virtual Machine (KVM).}
{\tiny  \item \textbf{Agents and tools} running on hosts including VDSM, QEMU, and libvirt. These tools provide local management for virtual machines, networks and storage.}
{\tiny  \item \textbf{The Red Hat Enterprise Virtualization Manager}; a centralized management platform for the Red Hat Enterprise Virtualization Environment. It provides a graphical interface where you can view, provision and manage resources.}
{\tiny  \item \textbf{Storage domains} to hold virtual resources like virtual machiens, templates, ISOs.}
{\tiny  \item A \textbf{database} to track the state of and changes to the environment.}
{\tiny  \item Access to an external \textbf{Directory Server} to provide users and authentication.}
{\tiny  \item \textbf{Networking} to link the environment together. This includes physical network links, and logical networks.}
\end{itemize}
\end{displayquote}

\section{De-Chroming the Acer c720 Chromebook}
This guide explains how to remove Google's ChromeOS from a Chromebook and to install Debian on that Chromebook.
\begin{flushleft}
\url{https://www.softwarefreedom.org/blog/2014/dec/01/de-chrome-acer-c720/}
\end{flushleft}
\begin{quotation}
\textit{By Ian Sullivan | December 1, 2014}

\textbf{What is De-Chroming?}

\textit{This talk is an instructional companion to the SFLC @ 10 Disposable Computing talk.}

De-Chroming is the process of taking a Chromebook laptop, in this case the Acer c720, and replacing the Chrome operating system with a full-featured Debian install.
\end{quotation}

\section{The Apple II User's Guide, Third Edition}
\linebreak
\begin{flushleft}
by Lon Poole with Martin McNiff and Steven Cook\par
Osborne McGraw-Hill\par
Berkeley, California\par
1985\par
\end{flushleft}
\linebreak
\begin{flushleft}
First paragraph of the introduction:
\end{flushleft}
\linebreak
\begin{displayquote}
\textit{The Apple II User's Guide, Third Edition} is your guide to the Apple II computer. It describes the Apple II itself along with the common accessories wuch as disk drives and printers. For those who aren't interested in programming the computer themsleves, the book explains how to use programs that can be bought off the shelf. For those who do want to learn how to write their own programs, the book provides detailed lessons with lots of examples.
\end{displayquote}
\linebreak

\begin{flushleft}
First paragraph of the first chapter, \textit{Presenting the Apple II}:
\end{flushleft}

\linebreak
\begin{displayquote}
A complete Apple II computer system includes several separate pieces of equipment. Figure 1-1 shows a typical system, centered around an Apple IIe. Your system may not look exactly like hte one pictured, because system components come from a long list of optional equipment. But there are three components that every system has in common: the Apple II computer itself, the built-in keyboard, and a television monitor. Let's take a closer look at each of these and at some of the more common pieces of optional equipment. This chapter does not explain how to hook up any of these components to the Apple II. For complete installation instructions, refer to the owner's manual supplied with each individual piece of equipment.
\end{displayquote}

\section{Now That You Know Apple Assembly \\ Language, What Can You Do With it?}

\begin{flushleft}
Jules H. Gilder
\end{flushleft}

\begin{flushleft}
Preface:
\end{flushleft}
\begin{displayquote}
This book is designed to be used by the newcomer to assembly language programming, who has already spent the time required to learn assembly language programming for the 6502 microprocessor and is now anxious to put his or her new found knowledge to work.
\end{displayquote}

\begin{flushleft}
Chapter 1 - Before You Get Started
\end{flushleft}

\begin{displayquote}
The 6502 microprocessor is probably the most widely used microprocessor in personal computers. It is found in the Apple II and Apple III families of computers, the PET, CBM, and VIC computers from Commodore, the Atari 400 and Atari 800 computers, and a variety of other computers and video games. Because of the popularity of the 6502, many books have been written on how to program in 6502 assembly language.
\end{displayquote}

\section{Applesoft II Basic Programming Reference Manual}
\begin{flushleft}
Overview\par
Introduction
\end{flushleft}
\begin{displayquote}
Applesoft II Basic is Apple's very much extended BASIC language. BASIC has been extended because there are many features on the APPLE II computer that just aren't available on other computers that use BASIC. By adding a few new words to the BASIC language, these features are immediately available to anyone using APPLESOFT. Among the feature supported by APPLESOFT are APPLE's color graphics, high-resolution color graphics and the direct analog inputs (the game controllers).
\end{displayquote}

\end{document}
