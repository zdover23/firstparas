\documentclass[a4paper,12pt]{article}
\usepackage{csquotes}
\usepackage{hyperref}
\title{First Paragraphs of Technical Documents}
\author{Zac Dover}

\begin{document}
\maketitle

\section{Upgrading and Repairing PCs}
\underline{Upgrading and Repairing PCs} by Scott Mueller begins with a paragraph consisting of two sentences. The first sentence broadly refers to the history of the development of the personal computer. The second sentence declares an intention to examine a few of those developments in order to give the reader a basis for understanding that history.
\begin{quotation}
Many discoveries and inventions have directly and indirectly contributed to the development of the PC and other personal computers as we know them today. Examining a few important developmental landmarks can help bring the entire picture into focus.
\end{quotation}

\section{Red Hat Enterprise Virtualization 3.4 Administration Guide}
This guide begins with a list of the components of a RHEV 3.4 environment.
\begin{displayquote}
{\tiny A Red Hat Enterprise Virtualization environment consists of:}
\begin{itemize}
{\tiny  \item Virtual machine \textbf{hosts} using the Kernel-based Virtual Machine (KVM).}
{\tiny  \item \textbf{Agents and tools} running on hosts including VDSM, QEMU, and libvirt. These tools provide local management for virtual machines, networks and storage.}
{\tiny  \item \textbf{The Red Hat Enterprise Virtualization Manager}; a centralized management platform for the Red Hat Enterprise Virtualization Environment. It provides a graphical interface where you can view, provision and manage resources.}
{\tiny  \item \textbf{Storage domains} to hold virtual resources like virtual machiens, templates, ISOs.}
{\tiny  \item A \textbf{database} to track the state of and changes to the environment.}
{\tiny  \item Access to an external \textbf{Directory Server} to provide users and authentication.}
{\tiny  \item \textbf{Networking} to link the environment together. This includes physical network links, and logical networks.}
\end{itemize}
\end{displayquote}

\section{De-Chroming the Acer c720 Chromebook}
This guide explains how to remove Google's ChromeOS from a Chromebook and to install Debian on that Chromebook.
\url{https://www.softwarefreedom.org/blog/2014/dec/01/de-chrome-acer-c720/}
\begin{quotation}
\textit{By Ian Sullivan | December 1, 2014}

\textbf{What is De-Chroming?}

\textit{This talk is an instructional companion to the SFLC @ 10 Disposable Computing talk.}

De-Chroming is the process of taking a Chromebook laptop, in this case the Acer c720, and replacing the Chrome operating system with a full-featured Debian install.
\end{quotation}

\end{document}
